\documentclass{article}
\usepackage{amsmath}
%\usepackage{geometry}
\usepackage[utf8]{inputenc}


\setlength{\parindent}{0pt}
\setlength{\parskip}{1em}

\begin{document}

\section{Power of point}
The power of point $P$ with respect to circle with centre $O$ and radius $r$ is defined as
$$ p= PO^2-r^2 $$

\textbf{Theorem 1: } 
For any line through point $P$ which intersects the circle at points $A$ and $B$ 
$$p= PA \times PB$$

\textbf{Theorem 2: }
For tangent $PC$ where $C$ is tangent point
$$p=PC^2$$  

\begin{enumerate}
\item 
Square $ABCD$ of side length $a$ has a circle inscribed in it. Let $M$ be the midpoint of $AB$ Find the length of that portion of the segment $MC$ that lies outside of the circle.

\item  % Prasolov 3.11
Line $OA$ is tangent to a circle at point $A$ and chord $BC$ is parallel to $OA$. Lines	$OB$ and $OC$ intersect the circle for the second time at points $K$ and $L$, respectively. Prove
that line $KL$ divides segment $OA$ in halves.

\item % LP 2003 XI-3
We have a triangle $ABC$. Points $K$, $L$ and $M$ are chosen on the sides $BC$, $AC$ and $AB$, such that $AK$, $BL$ and $CM$ intersect in one point. We know that $ALKB$ and $BMLC$ are cyclic quadrilaterals. Show that $AMKC$ is also a cyclic quadrilateral.

\item % Prasolov 3.12
On the longer diagonal $AC$ of parallelogram $ABCD$ point $M$ is chosen, such that $BCDM$ is a cyclic quadrilateral. Show that $BD$ is tangent to circumcircles of triangles $AMD$ and $AMB$.

\item % IMO shortlist 2013
Let $ABC$ be a triangle with $\angle B > \angle C$. Let $P$ and $Q$ be two different points on line $AC$ such that $\angle PBA = \angle QBA = \angle ACB$ and $A$ is located between $P$ and $C$. Suppose that there exists an interior point $D$ of segment $BQ$ for which $PD=PB$. Let the ray $AD$ intersect the circle $ABC$ at $R \ne A$. Prove that $QB = QR$.
\end{enumerate}


\newpage
\section{Radical axis}
Radical axis is the locus of points at which tangents drawn to both circles are equal.

\textbf{Theorem 1: }
The power of points on radical axis is equal with respect to both circles.

\textbf{Theorem 2: }
Radical axis is a line.

\textbf{Theorem 3: }
The three radical axes for three circles intersect in one point called the radical centre. 

\begin{enumerate}

\item %Prasolov 3.58a
Prove that the midpoints of the four common tangents to two non-intersecting circles lie on one line.

\item %puutujakuusnurga diagonaalid Prasolov 3.66
Prove that the diagonals $AD$, $BE$ and $CF$ of circumscribed hexagon $ABCDEF$ intersect in one point. (Brianchon theorem)

\item % TVV	2016 6
Circles $k_1$ and $k_2$ intersect at points  $M$ and $N$. Line $l$ intersects circle $k_1$ at points $A$ and $C$ and circle $k_2$ at points $B$ and $D$, such that points $A$,  $B$, $C$ and $D$ lie on the line $l$ in that order. Let $X$ be such point on line $MN$ that $M$ lies between $X$ and $N$. Rays $AX$ and $BM$ intersect at point $P$, rays $DX$ and $CM$ at point $Q$. Prove that $PQ \parallel l$.

\item % BT treening 2009-5
Point $E$ is chosen on the median $CD$ of triangle $ABC$. Line $AB$ is tangent to circle $c_1$ at point $A$ and to circle $c_2$ at point $B$ such that both circles go through point $E$. The second intersection of $c_1$ and  $AC$ is $M$. The second intersection of $c_1$ and $BC$ is $N$. Prove that tangent lines to circles $c_1$ and $c_2$ at points $M$ and $N$ respectively intersect on line $CD$.


\item % 3.6l
Three circles intersect pairwise at points $A_1$ and $A_2$, $B_1$ and $B_2$, $C_1$ and $C_2$. Prove that $A_1B_2 \times B_1C_2 \times C_1A_2 = A_2B_1 \times B_2C_1 \times C_2A_1$.


\item % Prasolov 3.60
The extensions of sides $AB$ and $CD$ of quadrilateral $ABCD$ meet at point $F$ and
the extensions of sides $BC$ and $AD$ meet at point $E$. Prove that the circles with diameters
$AC$, $BD$ and $EF$ have a common radical axis and the orthocenters of triangles $ABE$, $CDE$,
$ADF$ and $BCF$ lie on it.

\end{enumerate}

\newpage
\section*{Hints}

\textbf{Power of point}

\begin{enumerate}
	\item Write down power of point $C$.
	\item Identify similar triangles and write down power of the intersection of $KL$ and $AB$
	\item Write down the power of the intersection for each circle
	\item Write down the power of the intersection of the diagonals
	\item Prove that quadrilateral $DRCQ$ is cyclic
\end{enumerate}

 \noindent
 \textbf{Radical axis}

\begin{enumerate}
	\item 
	\item Find three circles for which the diagonals are radical lines for.
	\item What's the power of $X$ with respect to each circle? What other points lie on the circle $PQM$?
	\item Prove that quadrilateral $ABNM$ is cyclic.
	\item Where's the radical centre? Find 3 pairs of similar triangles.
	\item For each triangle, what's the power of orthocentre with respect to each circles?
\end{enumerate}

\end{document}