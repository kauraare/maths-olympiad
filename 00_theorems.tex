\documentclass{article}
\usepackage{amsmath}
\usepackage{amsfonts}
\usepackage{geometry}
\usepackage[utf8]{inputenc}
\usepackage{enumitem}
\usepackage{physics}
\setlength{\parindent}{0pt}
\setlength{\parskip}{1em}
\relpenalty=10000
\binoppenalty=10000

\begin{document}
\section{Geometry}
  \subsection{Inscribed angles}
    \begin{enumerate}
      \item Inscribed angles subtending the same arc are equal to each other
      \item Inscribed angle is equal to half of the central angle subtending the same arc.
      \item \textbf{Alternate segment theorem}: Angle between chord and tangent is equal to inscribed angle which subtends to that chord.
    \end{enumerate}
  \subsection{Power of point}
    \begin{enumerate}
      \item The \textbf{power of point} $P$ with respect to circle with centre $O$ and radius $r$ is defined as
      $$ p= PO^2-r^2 $$
      \item For any line through point $P$ which intersects the circle at points $A$ and $B$
      $$p= PA \times PB$$
    \end{enumerate}
  \subsection{Trigonometry}
    \begin{enumerate}
      \item Basic trigonometric identities
      $$\frac{\sin x}{\cos x} = \tan x$$
      $$\sin^2 x + \cos^2 x =1$$
      \item Double angle formulae
      $$ \cos 2x = \cos^2 x - \sin^2 x $$
      $$ \sin 2x = 2\sin x \cos x $$
      \item Sine law
      $$\frac{a}{\sin\alpha} = \frac{b}{\sin\beta} = \frac{c}{\sin\gamma} = 2R$$
      \item Cosine law
      $$ a^2 + b^2 - c^2 -2ab\cos\gamma = 0$$
    \end{enumerate}
\newpage
\section{Algebra}
  \subsection{Polynomials}
    \begin{enumerate}
      \item \textbf{Bezout's theorem}: A polynomial $P(x)$ is divisible by the binomial $(x-a)$ if and only if $P(a)=0$.

      \item \textbf{The fundamental theorem of algebra}: Every non-constant polynomial has a complex root.

      \item \textbf{The rational root theorem}: If $x = p/q$ is a rational zero
      of a polynomial $P(x) = a_nx^n +\hdots + a_0$ with integer coefficients and $p,q=1$,
      then $p | a_0$ and $q | a_n$.

      \item \textbf{Vieta's formulae}: If the solutions polynomial of degree $n$ are $x_1,x_2,\dots,x_n$ and  $a_n=1$, then the following holds:
      \begin{eqnarray*}
      x_1+x_2+\ldots+x_n &=& -a_{n-1},\\
      x_1x_2+x_1x_3+\ldots+x_{n-1}x_{n} &=& \hphantom{-} a_{n-2}, \\
      x_1x_2x_3+x_1x_2x_4+\ldots+x_{n-2}x_{n-1}x_n &=& -a_{n-3},\\
      \ldots \\
      x_1x_2\ldots x_n &=& (-1)^n a_0.
      \end{eqnarray*}
    \end{enumerate}
  \subsection{Inequalities}
    \begin{enumerate}
    	\item \textbf{General mean inequality}:
    	The mean of order $p$ of positive real numbers $x_1,\dots,x_n$ is defined as:
    	$$M_p=
    	\begin{cases}
    	\left(\frac{x_1^p+ \dots + x_n^p}{n}\right)^{1/p} &\text{for } p \ne 0 \\
    	\sqrt[n]{x_1 \dots x_n}                           &\text{for } p=0
    	\end{cases}
    	$$
    	In particular
    	\begin{center}
    		\begin{tabular}{lcl}
    			Smallest element & $\min\{x_i\}$ & $M_{-\infty}$ \\
    			Harmonic mean & HM & $M_{-1}$ \\
    			Geometric mean & GM & $M_0$ \\
    			Arithmetic mean & AM & $M_1$ \\
    			Quadratic mean & QM & $M_2$ \\
    			Largest element & $\max\{x_i\}$ & $M_{\infty}$
    		\end{tabular}
    	\end{center}
    	Then for any real $p$ and $q$
    	$$M_p \leq M_q \iff p \leq q $$

    	\item \textbf{Cauchy inequality}:
    	For real numbers $x_1, \dots , x_n, y_1, \dots , y_n$

    	$$\left(\sum_{i=1}^{n} x_i y_i\right)^2 \leq \sum_{i=1}^{n} x_i^2 \sum_{i=1}^{n} y_i^2 $$

    	\item \textbf{Chebyshev inequality}:
    	For real numbers $x_1 \geq \dots \geq x_n$ and $y_1 \geq \dots \geq y_n$
    	$$\frac{1}{n} \sum_{i=1}^{n} x_iy_i
    	\geq
    	\left(\frac{1}{n}\sum_{i=1}^{n}x_i\right)
    	\left(\frac{1}{n}\sum_{i=1}^{n}y_i\right)
    	\geq
    	\frac{1}{n} \sum_{i=1}^{n} x_iy_{n+1-i} $$

    	\item \textbf{Jensen inequality.}
    	Given positive real numbers $\lambda_1,\hdots,\lambda_n$ for which $\lambda_1+\hdots+\lambda_n=1$  and a convex function $f(x)$ the following holds:
    	$$f(\lambda_1 x_1 + \hdots + \lambda_n x_n) \leq \lambda_1 f(x_1) + \hdots + \lambda_n f(x_n)$$
    	Similarly, when $f(x)$ is a concave function, then
    	$$f(\lambda_1 x_1 + \hdots + \lambda_n x_n) \geq \lambda_1 f(x_1) + \hdots + \lambda_n f(x_n)$$
  	\end{enumerate}
  \subsection{Functional equations}
    \begin{enumerate}
      \item
      If $f(x)=f(y)$ implies $x=y$, then $f$ is \textbf{injective}
      \item
      If for each element $y$ in function codomain, there exists $x$ for which $f(x)=y$, then $f$ is \textbf{surjective}.
      \item
      If $f$ is both injective and surjective then $f$ is \textbf{bijective} (one-to-one).
      \item \textbf{Cauchy functions}: If any of the following is satisfied
      \begin{enumerate}
        \item The function is continuous at one point,
        \item The function is monotonic on any interval,
        \item The function is bounded on any interval.
      \end{enumerate}
      then all of the following functional equations have the respective solutions.
      \begin{align}
         f(x+y) &= f(x) + f(x) & \Rightarrow && f(x) &= cx \\
         f(xy)  &= f(x)f(y)    & \Rightarrow && f(x) &= x^c \\
         f(xy)  &= f(x) +f(y)  & \Rightarrow && f(x) &= c \log |x| \\
         f(x+y) &= f(x)f(y)    & \Rightarrow && f(x) &= e^{cx}
      \end{align}
    \end{enumerate}
\newpage
\section{Number Theory}
  \subsection{Divisibility}
    \begin{enumerate}
      \item
      If $a \mid b$ and $ c \mid d$ then $ac \mid bd$
      \item
      If $a \mid b$ and $a \mid c$, then $a \mid b+c$
      \item
      \textbf{Euclid's algorithm.} \\
      $\gcd (a,b) = \gcd (a,b-a)$
      \item
      \textbf{Corollary of Euclid's algorithm.} \\
      $ax+by=n$ has solution $(x,y)$ in integers if and only if $gcd(a,b) \mid n$
    \end{enumerate}
  \subsection{Congruences}
    For integers $a,b,c,d,m,n$ and prime $p$.
    \begin{enumerate}
      \item
      $a \equiv b \mod m \iff m \mid a-b $
      \item
      $a\equiv b \mod m$ and $ c\equiv d \mod m \implies a+c\equiv b+d \mod m$
      \item
      $a \equiv b \mod m \iff an \equiv bn \mod mn$
      \item
      $a \equiv b \mod m \implies an \equiv bn \mod m$
    \end{enumerate}
  \subsection{Exponential congruences}
    \begin{enumerate}
    \item
    \textbf{Fermat's little theorem.} \\
    For prime $p$ and integer $a$
    $$a^p \equiv a \mod p$$

    \item
    \textbf{Wilson's theorem.} \\
    $$(p-1)! \equiv -1 \mod p$$
    if and only if when $p$ is prime number.

    \item
    \textbf{Number of factors} \\
    The number of positive factors of $n=p_1^{\alpha_1} \dots p_k^{a_k}$
    $$d(n)= (\alpha_1+1)(\alpha_2+1)\dots (\alpha_k+1)$$
    \item
    \textbf{Sum of factors} \\
    The sum of positive factors of $n=p_1^{\alpha_1} \dots p_k^{a_k}$

    $$\sigma(n)= \frac{p_1^{\alpha_1+1}-1}{p_1-1} \dots  \frac{p_k^{\alpha_k+1}-1}{p_k-1}$$

    \item
    \textbf{Euler's function} \\
    Euler’s function or totient function $\varphi(n)$ is defined for $n=p_1^{\alpha_1} \dots p_k^{a_k}$ as the number
    of positive integers less than $n$ and coprime to $n$. Then
    $$\varphi(n) = n \left(1-\frac{1}{p_1}\right) \dots  \left(1-\frac{1}{p_k}\right)$$

    \item
    \textbf{Euler's theorem}  (Generalisation of Fermat's theorem)\\
    Let $n$ be a natural number and $a$ an integer such that $\gcd(a,n)=1$. Then
    $$a^{\varphi(n)} \equiv 1 \mod n $$
    \end{enumerate}
\newpage
\section{Combinatorics}
  \subsection{Counting of objects}
    \begin{enumerate}
      \item \textbf{Permutations}: $P_n=n!$
      \item \textbf{Variations}: $V^k_n=\frac{n!}{(n-k)!}$
      \item \textbf{Combinations}: $C^k_n=\frac{n!}{(n-k)!}$
    \end{enumerate}
  \subsection{Pigeonhole principle}
    \begin{enumerate}
      \item If a set of $nk + 1$ different elements is partitioned into $n$ mutually disjoint subsets, then at least one subset will contain at least $k + 1$ elements.
    \end{enumerate}
  \subsection{Graph Theory}
    \begin{enumerate}
      \item \textbf{Tree} is a connected graph with no circuits. A connected graph with $n$ vertices is a tree if and only if it has $n-1$ edges.
      \item \textbf{Euler path} is a path in which every edge of the graph appears exactly once. Likewise \textbf{Euler circuit} is a circuit in which every edge appears exactly once.
      \item If each vertex in a connected graph has even degree, then the graph contains an Euler circuit.
      \item If a connected graph has exactly two vertices with odd degree, it contains an Euler path.
      \item A \textbf{Hamilton circuit} is a circuit in which each vertex appears exactly once.
      \item
      A \textbf{planar graph} can be embedded in a plane with edges corresponding to non-intersecting lines (not necessarily straight). A planar graph with $n$ vertices has at most $3n-6$ edges.
      \item \textbf{Dirac's theorem}: A graph with $n$ vertices contains a Hamilton cycle if the degree of each vertex is at least $n/2$.
      \item \textbf{Euler's formula}: $E+2=F+V$, where $E,F,V$ are the numbers of edges, faces and vertices of a polyhedron.

  \end{enumerate}
\end{document}s
